% Start of header.
\documentclass{article}
\usepackage{lmodern}
\usepackage[T1]{fontenc}
\usepackage{amsmath}
\usepackage[utf8]{inputenc}
\usepackage{amssymb}
\usepackage{caption}
\usepackage{amsthm}

\newcommand{\I}[0]{\mathbb{I}}
\newcommand{\Q}[0]{\mathbb{Q}}
\newcommand{\Z}[0]{\mathbb{Z}}
\newcommand{\C}[0]{\mathbb{C}}
\newcommand{\F}[0]{\mathbb{F}}
\newcommand{\R}[0]{\mathbb{R}}
\newcommand{\N}[0]{\mathbb{N}}
\newcommand{\unif}[0]{\mathcal{U}}
\newcommand{\Tor}[0]{\operatorname{Tor}}
\newcommand{\ra}[0]{\rightarrow}
\newcommand{\rr}[0]{\Rightarrow}
\newcommand{\llrr}[0]{\Leftrightarrow}
\newcommand{\subq}[0]{\subseteq}
\newcommand{\sub}[0]{\subset}
\newcommand{\nsubq}[0]{\not\subseteq}
\newcommand{\nsub}[0]{\not\subset}
\newcommand{\sep}[0]{\hbox{ }}
\newcommand{\inv}[0]{^{\raisebox{.2ex,$\scriptscriptstyle-1$}}}
\newcommand{\thnew}[0]{^{\text{th}}}
\newcommand{\ifnew}[0]{\text{ if }}
\newcommand{\elsenew}[0]{\text{ else }}
\newcommand{\for}[0]{\text{ for }}

\setlength{\jot}{8pt}
% End of header
% Start of body.

\begin{document}

\section{Expected value of kth element of an ordered sequence of elements from some interval}

\subsection{Question:}

Let each ordered sequence of $n$ unique elements $X_1, \dots, X_n$ from the interval $(a, b)$ be equally probable. What is the expected value of the element $X_k$?

\subsection{Answer:}

Consider random variables $Y_1, \dots, Y_n$ sampled from a uniform  distribution over $(a, b)$ such that no two $Y_i$ are equal. First, we show that any ordered sequence made from $Y_1, \dots, Y_n$ is equally probable.

Let $X_1 = Y_{i_1}, \dots, X_n = Y_{i_n}$ be the ordered sequence made from our random variables $Y_1, \dots, Y_n$. Then we see the probability density function, $f$, at a specific random sequence is

$$f_X([X_1, \dots, X_n]) = \sum_{Y_1, \dots, Y_n \in (a, b)^n; Y_i \text{ unique}} P(Y_1, \dots, Y_n) $$
$$= \sum_{Y_1, \dots, Y_n \in (a, b)^n; Y_i \text{ unique}} f_{Y_1}(Y_1)\cdots f_{Y_n}(Y_n)$$

Note that the value of a uniform probability density does not change if we introduce a finite number of holes, thus:

\begin{gather}
f_{Y_i}(Y_i) =  \begin{cases} 1,  & \text{if $Y_i \in (a, b)$ and $Y_i \neq Y_j$ for $j \neq i$} \\ 0, & \text{otherwise} \end{cases}
\end{gather}

To see this
\end{document}
% End of body.